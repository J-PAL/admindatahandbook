\hypertarget{imf}{%
\chapter{The Use of Administrative Data at the International Monetary Fund}\label{imf}}

\printchapterauthor{%
\begin{authorlist}
  Era Dabla-Norris (International Monetary Fund)  \\
  Federico J. Díez (International Monetary Fund)  \\
  Romain Duval (International Monetary Fund)  \\
\end{authorlist}}\authorfootnote{Dabla-Norris, Era, Federico J. Díez, and Romain Duval}{Era Dabla-Norris, Federico J. Díez, and Romain Duval}{imf}
\authortoc{Era Dabla-Norris, Federico J. Díez, Romain Duval}
\hrulefill

\hypertarget{summary-8}{%
\section{Summary}\label{summary-8}}

This chapter describes the use of administrative data at the International Monetary Fund (IMF, the Fund) in the context of its three main operations: macroeconomic surveillance\index{surveillance} and research, lending to member countries, and technical assistance\index{technical assistance} to build capacity\index{build capacity} in policymaking in member countries. The chapter notes how the Fund has a long-standing tradition of using administrative data in some activities, but the systematic use for monitoring economic developments in member countries and research is still in its infancy. This is partly because the use of administrative data for macroeconomic analysis remains relatively recent, is resource-intensive, and often comes with strings attached. Several examples of interactions with government authorities (the ultimate providers of administrative data) in the context of specific projects are provided; these show wide variation in the degree of collaboration with national authorities, as well as in the procedures and (legal and technical) constraints in accessing and using the data. These country examples also highlight the challenges and opportunities, some of them unique to the institution, for IMF staff to obtain and work with such data. On the legal front, for example, while traditional judicial enforcement mechanisms for data use agreements are not applicable, since the IMF as an international organization is immune from the judicial process, potential partners that provide administrative data are comforted by other features of the IMF's immunities that provide strong protections to confidential information---and staff can and do negotiate one-off data access agreements with national authorities. The IMF's specificities also show in how it can ensure safe projects, people, settings, data, and outputs. For example, staff can leverage the institution's strong credibility, infrastructure, and procedures along these safe dimensions, but there is still wide variation in practices depending on individual member countries' requirements, and overall success in accessing and using administrative data is easier to achieve in a technical assistance context where demand for IMF analysis and research originates from the source country itself. In the future, through its bilateral engagement with its 189 member countries, participation in international data initiatives, and partnerships with universities and research networks, the IMF has the potential to gradually enhance the comparability, access, and use of (selected) administrative data produced by national authorities.

\hypertarget{introduction-9}{%
\section{Introduction}\label{introduction-9}}

\hypertarget{motivation-and-background-7}{%
\subsection{Motivation and Background}\label{motivation-and-background-7}}

The IMF, as an international organization, works closely with the country authorities of its membership. These include tax administrations, central banks, and financial supervisory authorities that generate administrative data of immense value. As such, the Fund has potential access to a wealth of administrative data through its engagement with 189 member countries.\footnote{The Fund has several frameworks in place regarding data. Under Article VIII, Section 5 of its articles of agreement, certain data must be provided to the Fund. Under the Data Standards Initiatives, members voluntarily subscribe to certain standards for data publication. This chapter refers solely to data that are voluntarily provided to the Fund and that fall outside of these other frameworks.} However, overcoming the financial, institutional, and technical hurdles to access administrative data remains a key challenge to unlocking the potential for research purposes. Building capacity\index{Building capacity} to utilize administrative data at the IMF and within member countries would enable the IMF to better answer a series of long-standing research questions in macroeconomic, financial, and structural arenas.

The IMF's interactions with members involve three main activities. First, the IMF conducts macroeconomic surveillance\index{surveillance}, which entails monitoring of economic and financial developments in each member country and the provision of policy advice. Second, the IMF provides financing to members with balance of payments problems. Finally, the IMF provides technical assistance\index{technical assistance} and capacity development\index{capacity development} programs in its areas of expertise.\footnote{While the surveillance and policy recommendations are normally conducted annually and for every member country, IMF financing and technical assistance are demand-driven and can only be initiated at the request of country authorities.} In addition, the Fund's work in economic research and statistics supports all three of these activities. Each of these operations can potentially lead to IMF staff accessing and working with administrative data owned by a national government. As such, the Fund has a long-standing tradition in the use of administrative data that is tightly linked to its core functions. However, the intensity with which this kind of data are used varies significantly across different lines of Fund operations.

The use of administrative data is most common in the case of technical assistance\index{technical assistance} requested by agencies or instrumentalities within member countries. For example, when analyzing the impact of tax reforms or weaknesses in existing tax systems, social programs, or banking sector reforms, Fund staff are likely to have access to confidential administrative data (e.g., tax records for specific taxes, government spending programs, or credit registry and bank loan-level data) to assess the impact of the policies being evaluated. Technical assistance missions on statistics often work with authorities to develop the use of granular administrative data alongside or in place of survey data (e.g., the use of tax data to estimate GDP by production \citep{rivas2018}); these data can then be used by IMF staff for policy evaluation. Similarly, in the context of provision of IMF financing, country authorities often make confidential administrative data available to Fund staff to better assess program performance and compliance with conditionality linked to IMF financing or to assess and evaluate the impact of policies that are intended to affect certain groups or activities.

There is growing recognition within the IMF that the systematic use of administrative data could also be beneficial for macroeconomic surveillance\index{surveillance} and research, a key pillar of its activities. This could allow for quicker and better-targeted policy responses to changes in business conditions and more granular assessment of member country policies. For example, using administrative data would enable IMF staff to better assess the implications for consumption and income distribution of alternative tax policy recommendations in a country.

The systematic use of administrative data, however, is less common in the context of IMF surveillance\index{surveillance} and research than for the Fund's two other main activities. One reason is that the use of micro-administrative data for macroeconomic analysis (and surveillance) has only become a routinely used tool within the last ten years, even in academia. Another reason is that utilizing administrative data is resource-intensive both in terms of the direct financial costs involved and staff time required to access and process the data. Further, even with interest from the Fund, national authorities can be unwilling to share the data for surveillance and research purposes, as the benefits to them are not always clear-cut. In these instances, and lacking any systematic institutional protocols, obtaining data rests on personal efforts by IMF staff to approach and potentially partner with national authorities.\footnote{Moreover, it can be particularly challenging to develop and maintain longer-term personal relationships with national authorities since IMF staff tend to rotate jobs within the Fund.} Finally, in many instances, legal constraints and confidentiality concerns pose challenges to accessibility. For example, countries often require that IMF staff be physically present to access the data or grant access indirectly through collaboration with a local staffer.

\hypertarget{data-use-examples-7}{%
\subsection{Data use examples}\label{data-use-examples-7}}

This section provides concrete examples where Fund staff has used administrative data for policy and surveillance\index{surveillance} work that have broader relevance for the research community, highlighting the procedures followed for securing data access from authorities, the challenges faced by staff, and the outputs from the projects.

\hypertarget{marrying-research-and-technical-assistance-the-case-of-peru}{%
\subsubsection{Marrying research and technical assistance: the case of Peru}\label{marrying-research-and-technical-assistance-the-case-of-peru}}

Peru's electronic invoicing (e-invoicing) tax reform study offers an interesting example of how policy questions raised in the context of IMF's technical assistance\index{technical assistance} activities, and relationships built with country authorities in the process, were leveraged to develop research projects using administrative data. The Peruvian government faced challenges with value-added tax (VAT) compliance as identified by the IMF's Revenue Administration's Gap Analysis Program (RA-GAP\index{RA-GAP}) Assessment.\footnote{The RA-GAP assessment provides a systematic evaluation of the revenue administration's operations for the VAT, assessing which group of taxpayers are contributing to tax gaps---the difference between potential and actual revenue collections---and identifies potential causes and sectoral gaps \citep{hutton2017}. The assessment itself requires detailed tax return records, tax payment and refund records, and customs data. This information is combined with tax registration information and detailed National Accounts data and input-output or source-use tables. The tax authority or the ministry of finance in the country shares the confidential administrative data with an IMF team electronically during a field visit, but in an anonymized manner.}

The 2015 RA-GAP\index{RA-GAP} analysis for Peru had highlighted weaknesses in VAT collections and pointed to the recent introduction of electronic invoicing (e-invoicing) as a potential tool for increasing revenue collections. The VAT is particularly susceptible to compliance risks arising from overclaiming of input tax credits by submission of false or altered invoices. By digitalizing transaction data, the presumption was that e-invoicing should allow for greater oversight and review of tax refund applications by the firm and its suppliers, increasing the probability of evasion detection and thereby encouraging greater voluntary compliance. A key question of interest to the Peruvian tax authority (SUNAT) was whether the e-invoicing reform had achieved its intended objectives and for which types of firms compliance had improved the most. From the IMF's perspective, this question was of broader relevance as a number of emerging market and developing economies have introduced e-invoicing in recent years, but there were few studies assessing its efficacy.

The electronic transmission of invoice information in Peru required a substantial overhaul of tax administration and taxpayer IT capabilities. Until 2014, the SUNAT promoted voluntary use of e-invoices, relying primarily on paper invoices. From 2015 onwards, a schedule for the mandatory incorporation of firms into the e-invoicing system was introduced: the first reform waves focused on larger firms and priority industries, while smaller firms were given more time to comply. This sequential introduction of the reform provided a useful empirical identification strategy. Moreover, assignment of firms to each group was based on a revenue threshold and other firm characteristics, which could allow for assessing the heterogenous reform effects. However, the reform evaluation itself fell outside the purview of the IMF's standard technical assistance\index{technical assistance} activities and required additional funding.

Around this time, the Bill \& Melinda Gates Foundation had entered into a partnership with the IMF's Fiscal Affairs Department to advance research and practice on using digital advances to improve public finance in emerging market and developing economies. Financial support was provided by the foundation to advance this agenda within the Fund and showcase this more broadly to the Fund's membership. The IMF research team outlined a proposal to use these funds to evaluate the impact of e-invoicing reform in Peru, exploiting the quasi-experimental variation in the reform rollout. The reform design allowed a precise comparison of firms that were already required to digitalize against similar ones that had yet to do so. In addition to specific findings for Peru, the project aimed to shed light on the implementation of digital technologies for tax administrations in emerging market and developing economies more broadly. The team's proposal identified how the study could provide insights for IMF technical assistance\index{technical assistance} on the efficacy of e-invoicing mechanisms and their interaction with other tax compliance tools, which could feed into the design of e-invoicing roll-outs by other countries. The project was approved by an internal IMF committee and funding was provided for travel by the research team to organize joint workshops and conferences with the tax authority.

To evaluate the reform, the IMF team needed to draw from the comprehensive data set of all firms' monthly tax reports in Peru. Due to the sensitive nature of taxpayer data, it was clear at the outset that the team would not have direct access to the data, and the project itself would require buy-in from SUNAT. The IMF's strong relationship with SUNAT, thanks to ongoing provision of technical assistance\index{technical assistance}, led not only to the endorsement of the study by the tax authority but also a commitment by SUNAT on a close collaboration to advance the empirical analysis.

The IMF team agreed with SUNAT to design the analysis and work remotely. The size and confidential nature of the data called for an innovative collaboration: the IMF put together a team of experts in different fields and from the Inter-American Development Bank (IDB) and different departments within the IDB who had used similar data to discuss the broader policy questions and work strategy. The database remained on servers in Peru while the work was conducted across countries via remote communication. The IMF team relied on remote processing and sent scripts to a staff member designated by SUNAT as the \emph{point-person} within the tax authority. This point-person had some in-house skills to run the scripts, but this was complemented by a concerted effort from the IMF research team to provide coaching and guidance on econometric packages. To facilitate the process, the IMF team constructed mock databases to test and troubleshoot the scripts.

It was also clear from the outset that exclusively remote working would not suffice to advance the project. Inevitably, the project would require on-site work to refine the analysis and resolve potential hurdles encountered. As a result, the funding request has explicitly budgeted several short visits to Peru for fieldwork and discussions with counterparts in the tax authority. The engagement with SUNAT was further strengthened through workshops and joint seminars to discuss initial findings, share know-how on working with administrative data, and solicit inputs from various stakeholders, which culminated in a co-authored IMF working paper with the authorities \citep{bellon2019}. The understanding with SUNAT was that the paper would be subsequently revised by the IMF team for journal publication (as of this writing, the \emph{Journal of Public Economics} requested the IMF to revise and resubmit the paper).

SUNAT found the interaction to be useful for presenting findings to ministerial level decision-makers as well as highlighting the value of the research conducted by the tax administration. Given the success of the first-round engagement, SUNAT was eager to further exploit the value of administrative data for research. The initial capacity building\index{capacity building} investment by IMF has led to repeated engagements and to the Peruvian tax administration to be more open to outside researchers also utilizing the tax administration's data, using similar data sharing and use protocols as for the IMF team. An ongoing follow-up project uses transaction-level data to examine and quantify spillovers in technology adoption by different types of firms \citep{holtsmark2020}.

\hypertarget{enhancing-surveillance-the-case-of-vietnam}{%
\subsubsection{Enhancing surveillance: The Case of Vietnam}\label{enhancing-surveillance-the-case-of-vietnam}}

Ongoing projects with the General Statistics Office (GSO) of Vietnam are examples of using administrative data to inform policy in the context of IMF's surveillance\index{surveillance} activities and fostering buy-in from the data provider for continued engagement. Census data and other labor and household surveys are confidential in Vietnam and not easily made accessible to researchers or even IMF staff. The IMF country team on Vietnam wanted to examine corporate vulnerabilities arising from the COVID-19 pandemic, but obtaining access to the firm census data was not straightforward.

As a first step, the contours of the project were explicitly outlined in a letter to the highest level of the GSO, and assurances provided that data would only be used for their intended purpose. It was understood that the Washington D.C.--based IMF team would not have direct access to the census data and would need to work remotely with a designated point-person in the GSO. One challenge was that the GSO office hosting the census data was not well-versed in different statistical packages, particularly Stata. The IMF team would have to hire someone locally to provide on-site training on Stata before the work could begin, but funding for this type of research activity is not readily available in the IMF. After much internal deliberation, a small pot of discretionary funds was found. This was used by the IMF team to hire a Vietnamese university professor known to GSO staff. The professor provided on-site training and served as facilitator between the two teams, helping with the running of scripts prepared by the IMF team.

The GSO team initially shared a 10 percent anonymized sample from the firm census that served as the basis for developing the scripts and agreed to share detailed summary statistics and moments of the data requested by the IMF. The IMF team worked closely with the GSO, sharing scripts, receiving summary tables and statistics, and exchanging views on the results.\footnote{In other cases, access to the administrative data occurred without IMF staff working along with authorities on a joint project. For instance, Fund researchers have utilized both the confidential social security and economic census data from Mexico. For the economic census data, IMF researchers had access to the underlying confidential survey data, but this required travel to Mexico to access the data. Brazil, Canada, Colombia, Denmark, France, Luxembourg, Slovakia, and South Africa are examples of other countries where the IMF has had access to administrative data for research purposes.} This initial engagement culminated in a policy note that was jointly co-authored with staff from the GSO and widely shared within the Vietnamese government, including at the ministerial level. The publicity garnered by the project and the close working relationship fostered trust and resulted in buy-in from the GSO on continued future engagement.\footnote{An ongoing project with the Norwegian authorities is another example of using administrative data for joint research to inform policy in the context of IMF's surveillance activities. Using novel administrative data covering the universe of registered electric cars combined with detailed information of the owners, the project with the Norwegian authorities analyzes the environmental effects, economic costs, and distributional consequences of electric cars \citep{holtsmark2020}.} The IMF team and the GSO are in the process of organizing seminars to more widely disseminate the results of their joint work in Vietnam and have outlined a series of future projects that will shed light on policy questions of relevance to the Vietnamese government.

\hypertarget{legal-framework}{%
\section{Legal Framework}\label{legal-framework}}

Some of the main challenges to accessing administrative data include understanding the legal frameworks regarding the usage (and potentially transfer) of the data and overcoming any associated confidentiality issues. IMF staff face most of the constraints that other users of administrative data must overcome, as well as several issues unique to the IMF.

As mentioned above, IMF staff working on technical assistance\index{technical assistance} projects and IMF-supported programs in member countries often require administrative data to perform the necessary analyses. In these instances, the country usually has a straightforward process for making the data available to the IMF. Often, the IMF's existing confidentiality framework---grounded in the articles of agreement\index{articles of agreement} by which all members agree to abide---is sufficient to address any domestic legal requirements on sharing information. However, as in the case of engagement with the statistical authority in Vietnam, the Fund may also agree that the data will be used only for stated purposes and not for any other research activities.

In some cases, however, the IMF may also wish to perform follow-up research utilizing the data that are outside of the scope of the original activities, for instance, to generalize lessons learned in one country across other countries. These projects require reaching additional agreements with the country authorities. The terms and sophistication of these agreements, as well as the authorities' willingness for further engagement, vary significantly across contexts.

A legal issue unique to the IMF and other similar organizations (such as the World Bank) is its status as an international organization with immunity from judicial process. All members of the IMF have committed to granting this immunity by signing the IMF's articles of agreement\index{articles of agreement} and enacting domestic laws that give effect to the privileges and immunities set forth in the IMF's articles of agreement.\footnote{In the United States, Section 11 of the Bretton Woods Agreements Act (public Law 171-79th Congress, 59 Statutes at Large, page 512 et seq., approved July 3, 1945, 22 U.S.C. Section 286h), gives full force and effect to the privileges and immunities of the IMF set out in its articles of agreement.} In practice, because the IMF's immunities, traditional judicial enforcement mechanisms are not applicable for data use agreements\index{data use agreement} with the IMF. Fund staff do not have the authority to waive the IMF's immunities in any agreement with a member or domestic agency to use its administrative data. While this unique characteristic may initially lead to reluctance on the part of potential partners to provide data to the IMF for research purposes, countries are usually comforted by the IMF's other immunities that provide strong protections to confidential information. For example, pursuant to the immunities of the Fund under Article IX of the Fund's articles of agreement, information and documents provided by members (or any other party) to the Fund form part of the Fund's archives, which are inviolable. ``Inviolability'' has been applied to mean that all non-public information or documents generated within or received by the Fund from members or other parties are protected by the Fund's immunities and would only be disclosed (including in response to a subpoena) with the approval of that member or other party and in accordance with the Fund's policies.

In addition to the data accessed through circumstances unique to the IMF, such as technical assistance\index{technical assistance} programs, the IMF also accesses administrative data for research purposes through existing mechanisms of data access or one-off agreements between the IMF and individual countries. In the Peru and Vietnam cases described above in the section on Data Use Examples, data access rested on one-off agreements that were subsequently extended to cover multiple projects. With the Mexican statistical agency (INEGI), IMF researchers can use existing access mechanisms that INEGI has established for academic and non-academic institutions to work with data on-site. In the framework of this agreement, researchers can establish individual bilateral arrangements with INEGI by completing a form listing the project details and the supervisor who has approved the project.

\hypertarget{making-data-usable}{%
\section{Making Data Usable}\label{making-data-usable}}

While most countries generate increasingly rich administrative data, making them readily available is challenging for various reasons. Since administrative data already exist and are usually collected on a regular or pre-set frequency, government agencies do not incur additional costs for data collection. However, country authorities may not have the proficiency and data management expertise to clean, anonymize, and organize the data into the format required for analysis. Often the issue stems from a lack of financial resources to build the technical capacity as seen in the Vietnam case. There is also the concern that confidential data could be de-anonymized and released into the public domain. Secondary disclosure (identification via deduction, particularly in concentrated markets) is also a concerning issue.

In some instances, the process of making data accessible is made easier by providing technical assistance\index{technical assistance} on how to extract and prepare the data as in the case for the VAT revenue gap\index{revenue gap} assessment. This revenue gap assessment itself is conducted by experts from the IMF and other international institutions working closely with a local team familiar with tax administration operations, tax design and policy, and statistical data. The first step in the technical assistance program is to identify available data and assess the quality, including by ensuring that data are accurately classified.\footnote{There is a key initial and ongoing role here for national statistical agency staff familiar with industry/product coding to ensure that data are accurately classified and routinely reviewed to avoid mismatches and double counting.} The simple task of reviewing the quality and scope of available tax record data can help tax administrations make the data more usable for their own analysis and in some instances, make data more research usable. As a second step, more harmonized tools and templates are used to extract the information required to conduct the tax gap analysis.

In some cases where data are made accessible, the IMF provides technical assistance\index{technical assistance} and training on data cleaning, for example, by employing an in-house developed algorithm. In other cases, IMF staff can spend considerable time cleaning and processing the data, particularly in instances where there is a lack of documentation, or the data already had been prepared for other purposes. This can be challenging given the relatively short time horizons for many IMF projects, which results in IMF staff often not having time to invest in data sets requiring significant cleaning and manipulation before becoming operative.

The IMF is usually interested in conducting cross-country analysis in order to have broad-based evidence on which to base the recommendations provided to its members. However, using administrative data on a cross-country basis is extremely challenging, since countries collect data in different formats and with different measures of outcomes, resulting in the IMF not using administrative data for these purposes. These issues transcend any individual country's data collection methods and can only be (fully) addressed through international standards for data classification and comparability. One possible long-term path to addressing this issue would be to follow the approach the IMF has taken with its balance of payments manual that provides guidance on how to compile balance of payments and related data. A similar approach toward how countries collect, document, and disseminate administrative data would be a potential path forward in improving the use of administrative data.

The IMF together with the Financial Stability Board (FSB) in collaboration with the Inter-Agency Group on Economic and Financial Statistics (IAG) have been leading the work on the G20 Data Gaps Initiative (DGI) since 2009. A second phase of the DGI (DGI-2), which started in 2015, sets more specific objectives for G20 economies to compile and disseminate minimum common data sets. While assisting countries in implementing DGI-2 recommendations, the IMF has been underscoring the importance of administrative data.

In recent years, the IMF has been entering into partnerships with universities and research networks to expand the availability of cross-country, comparable data--- which can be used for policy-relevant analysis---for countries in specific regions. For instance, firm-level data for countries in Asia are not readily available for researchers and policymakers. This has resulted in a partnership with the Productivity Research Network (PRN) at the National University of Singapore. Through this collaboration PRN will allow IMF staff to access the PRN's firm-level data set, based on country-level industrial censuses, for all available countries, and to access the underlying scripts. In exchange, the IMF will support the expansion of country coverage by reaching out to relevant country authorities and statistical agencies, encouraging them under their privacy policy to work with PRN in the data collection exercise. The PRN will be allowed to mention the IMF as a partner along with other international financial institutions involved in the project, and the IMF will be at liberty to publish analytical pieces based on data compiled by PRN, along with the adequate attribution and referencing of data sources, without prior consent. Making use of such collaborations to access harmonized administrative data (and not just survey-based data) will be the next frontier.

\hypertarget{protection-of-sensitive-and-personal-data-the-five-safes-framework-7}{%
\section{Protection of Sensitive and Personal Data: The Five Safes Framework}\label{protection-of-sensitive-and-personal-data-the-five-safes-framework-7}}

\hypertarget{safe-projects-7}{%
\subsection{Safe Projects}\label{safe-projects-7}}

As noted in earlier sections, one of the common ways that the IMF gets access to administrative data is through providing technical assistance\index{technical assistance}. As these activities are undertaken at the request of agencies or instrumentalities within member countries, they are by definition considered safe projects by the member country.

Performing program evaluations can be more challenging. Projects that can highlight potential issues and weaknesses with member countries are often sensitive, requiring buy-in from bureaucratic and political leadership in member countries. Given the ad hoc nature of proposing these research programs, each member country has different circumstances and criteria for evaluating research proposals. The IMF is generally most successful when proposing research projects on issues that the member nation wants to address. In the Peru example, the tax administration was interested in evaluating the effectiveness of their revenue mobilization program. When the IMF proposed a research project that addressed the specific question, it was easy for IMF staff to find buy-in at the tax administration for the project. In the Vietnam experience, the research project was undertaken in collaboration with the country authorities in the context of the IMF surveillance\index{surveillance} activities. It was considered \emph{safe} as the project was part of IMF country engagement.

Given the prerequisites of having member nations sign-on to research projects and the lack of a systematic mechanism for having them evaluate prospective projects, it is incumbent on the IMF staff to take the first step on conveying the merits of projects to countries. The reliance on interpersonal relations and existing engagements with government officials to initiate projects also means that getting the projects off the ground rests on personal initiative. The difficulty in identifying partners and securing buy-in can also limit the topics and types of projects that IMF researchers attempt to pursue. In an effort to address these issues, there have been attempts at cataloging the institutional history with different partners to identify those that are willing to share data for research purposes.

\hypertarget{safe-people-7}{%
\subsection{Safe People}\label{safe-people-7}}

From the perspective of member countries, international organizations such as the IMF are perceived as \emph{safe}, although the determination of what type of staff should be given access to confidential data and for what purposes is heavily dependent on the specific member country. While IMF staff are always subject to general IMF policies on the protection of confidential information, when working on data through technical assistance\index{technical assistance} programs or research projects, IMF researchers may agree to additional confidentiality. In some instances, tax authorities might agree to provide detailed records, including data at the enterprise level, only to designated IMF staff and their managers.

Internally, the IMF maintains confidentiality requirements upon its staff to ensure that it maintains its reputation as a trusted advisor. Disciplinary measures that apply in case of breaches of security are clarified in internal protocols. Significant efforts are made to educate IMF staff around issues of data confidentiality, including IMF staff receiving training in information security and ethics as part of their general employee training. Related to this, the IMF also maintains an intranet for researchers that outlines data sets that are accessible to internal researchers and the requirements for access to each data set.

The IMF occasionally brings in outside researchers for projects on a contractual basis. When outside researchers are contracting with the IMF, they are covered under the same confidentiality understandings, rules, and procedures that IMF staff operate.

\hypertarget{safe-settings-7}{%
\subsection{Safe Settings}\label{safe-settings-7}}

Depending on each individual member country and specific project, requirements for accessing and storing data can vary. The access mechanisms span a range of options from traveling to a secure facility in the member country to being allowed to download the data via a secure connection to IMF servers. In the case of some technical assistance\index{technical assistance} missions, access to the data can be restricted to those given explicit authorization by the data authorities. The data can also be encrypted in such a way that even the IT personnel who service the server cannot see the data on it.

In the instances where the IMF is authorized to store the data, the data are placed on a secure server with access-controlled folders. IMF IT staff place access controls on the server to restrict data access to staff with proper approval. The IMF maintains this infrastructure for many of its technical assistance\index{technical assistance} missions, which require that relevant staff and researchers have ongoing regular access to data from the member country. Staff and researchers have access to a wide range of IT resources and statistical software (SAS, Stata, Python, and R, among others) to choose from for their analysis. Data providers are consulted on the software available to share common platforms for the analysis.

\hypertarget{safe-data-7}{%
\subsection{Safe Data}\label{safe-data-7}}

As with the other aspects of the five safes framework, the determination of what constitutes safe data is up to the member country for each project in which the IMF is involved. In some cases, the data can be highly sensitive. This is particularly the case for individual tax records, data on recipients of specific government spending programs, or bank loans to individual firms and households. More generally, any data on individual reporting units or specific transactions/instruments, which in most cases allow the identification of individual entities, are therefore considered confidential. In these cases, governments themselves often make modifications to the raw data before sharing with the IMF, with links that could be used to identify individuals or other entities typically stripped for security reasons.

To encourage the sharing of sensitive information and documents, IMF staff share the institution's data protocols that describe procedures aimed at preventing unauthorized access to, and disclosure of, sensitive information and documents obtained through the country engagement.

In some instances, the same data provider will have different access requirements based on different types of data. For example, in Brazil and Mexico, accessing firm-level data requires that researchers travel to a data center, while more aggregated data can be downloaded instead. Usually, staff directly involved in a given project will have access to the same data, but access to these data will be restricted to them and their managers. In contrast, other Fund staff would only see aggregated data.

\hypertarget{safe-outputs-7}{%
\subsection{Safe Outputs}\label{safe-outputs-7}}

The decision on what outputs are acceptable is mostly at the discretion of the member country. While the existence of most technical assistance\index{technical assistance} missions is public information, whether a country makes its diagnostic public is at its own discretion, even if it does not involve the use of administrative data. Under \href{https://www.imf.org/external/np/pp/eng/2013/061013.pdf}{IMF policy}, the recipient of technical assistance makes the decision of whether they would like to make the final advice public. Government agencies and instrumentalities are typically reluctant to make public the state of their government finances, banking system, or health public information. At the same time, the IMF encourages the wider dissemination and publication of technical assistance information.

In many cases, countries require that IMF researchers follow standard statistical disclosure precautions such as mandating a minimum number of observations when countries allow results to be published. Agencies in different countries also have varying levels of familiarity with research and publications, which can require that IMF staff spend additional effort in detailing the different instances of the publication process and preparing the output for their approval. In agencies where the supervisors and data analysts are also familiar with academic research, this process tends to be easier.

\hypertarget{sustainability-and-continued-success-7}{%
\section{Sustainability and Continued Success}\label{sustainability-and-continued-success-7}}

Any Fund project involving the use of administrative data is assessed in terms of its costs and its performance. In regard to the former, much of the cost involved in generating and sharing (or eventually transferring) administrative data to the IMF is borne by the member country. In low- and middle-income countries, it is often the case that the agencies that might provide data suffer from a lack of financial resources to hire sufficient data staff. From the IMF's standpoint, the main costs usually are in terms of staff's time and traveling costs, whenever data access requires physical presence in the country. The direct financial costs for accessing administrative data, such as fees, are relatively minor, especially compared to those incurred by the IMF to purchase large firm-level data sets.

Regarding performance, the metrics used vary significantly with the nature of the project as each of the IMF's core activities (macroeconomic surveillance\index{surveillance} and research, lending, and technical assistance\index{technical assistance}) entail different objectives. That said, the ultimate goal is to provide valuable (and in some cases, actionable) information to senior IMF and national authorities to help them develop appropriate policy recommendations and actions, respectively---advice that would not be as detailed if access to administrative data was lacking. Another metric of success is the instance of a given project leading to follow-up projects (as in the example of Peru and Vietnam). This indicates a confirmation by the authorities of the \emph{modus operandi}: continued engagement speaks to success in creating deeper linkages with the data providers. Successful engagement in some countries could also be shown as an example that encourages other data providers to offer access to their data.

\hypertarget{concluding-remarks-4}{%
\section{Concluding Remarks}\label{concluding-remarks-4}}

The intensity and frequency with which administrative data are used at the IMF depends on whether these are used for technical assistance\index{technical assistance}, lending-related purposes, or for macroeconomic surveillance\index{surveillance} and research. While the use of administrative data usage is common in the first two areas it is much less so in the latter. In the absence of institutional protocols and incentives, the use of such data for research purposes depends on staff initiative in approaching national authorities/data providers. There is an increasing recognition, however, of the benefits of using these kinds of data for Fund's surveillance and research leading to several successful engagements such as the ones described in this chapter. At the same time, the Fund's unique status as an international organization creates opportunities. In particular, in the future, through its bilateral engagement with its 189 member countries, participation in international data initiatives and partnerships with universities and research networks, the IMF has the potential to gradually enhance cross-country comparability, access, and use for (at least some) administrative data produced by national authorities for research purposes.

\hypertarget{about-the-authors-9}{%
\section*{About the Authors}\label{about-the-authors-9}}
\addcontentsline{toc}{section}{About the Authors}

\href{https://www.imf.org/external/np/cv/AuthorCV.aspx?AuthID=299}{Era Dabla-Norris} is the mission chief for Vietnam and division chief at the IMF's Asia-Pacific Department. Previously, she was a division chief at the* IMF's Fiscal Affairs Department.

\href{https://www.imf.org/external/np/cv/AuthorCV.aspx?AuthID=311}{Federico J. Díez} is an economist at the Structural Reforms Unit of the IMF's Research Department. Previously he was an economist at the Research Department of Federal Reserve Bank of Boston.

\href{https://www.imf.org/external/np/cv/AuthorCV.aspx?AuthID=291}{Romain Duval} is the Head of the Structural Reforms Unit and an Advisor to the Chief Economist in the IMF Research Department. Previously he was the Division Chief for Regional Studies of the IMF Asia-Pacific Department and, prior to joining the Fund, the Division Chief for Structural Surveillance at the OECD Economics Department.

\begin{invisible}
This is a workaround for citations in footnotes, please ignore.
@holtsmark2020 @hutton2017
\end{invisible}

\putbib

